
\chapter{제목 만들기}
새로운 장은 시작하는 가장 큰 제목인 챕터 제목은 다음과 같이 작성합니다.
\begin{verbatim}
\chapter{챕터 제목}
\end{verbatim}
내용 작성

\section{큰 제목}
큰 제목은 다음과 같이 작성합니다.
\begin{verbatim}
\section{큰 제목}
\end{verbatim}
내용 작성

\subsection{중간 제목}
중간 제목은 다음과 같이 작성합니다. 중간 제목까지 자동으로 생성되는 차례에 포함됩니다.
\begin{verbatim}
\subsection{중간 제목}
\end{verbatim}
내용 작성

\subsubsection{작은 제목}
작은 제목은 다음과 같이 작성합니다. 작은 제목 부터는 번호가 붙지 않습니다.
\begin{verbatim}
\subsubsection{작은 제목}
\end{verbatim}
내용 작성

\paragraph{더 작은 제목}
더 작은 제목은 다음과 같이 작성합니다. 더 작은 제목은 설명형 리스트와 유사합니다.
\begin{verbatim}
\paragraph{더 작은 제목}
\end{verbatim}
내용 작성

\subparagraph{가장 작은 제목}
가장 작은 제목은 다음과 같이 작성합니다. 더 작은 제목과 같지만, 약간 들여쓰기가 됩니다.
\begin{verbatim}
\subparagraph{가장 작은 제목}
\end{verbatim}
내용 작성
